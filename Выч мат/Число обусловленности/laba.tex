\documentclass[14pt, titlepage,fleqn]{extarticle}
\usepackage[T1,T2A]{fontenc}
\usepackage[utf8]{inputenc}

\usepackage{amsmath}
\usepackage[russian]{babel}

\usepackage{titlepage}
\usepackage[final]{pdfpages}
\usepackage{listings}
\usepackage{color}
\usepackage{graphicx}
\usepackage{float} 

\usepackage{caption}


\newcommand{\InsertGraf}[2]{
	\begin{figure}[H]
		\center{\includegraphics[width = 1\textwidth]{#1}}
		\caption{#2}
	\end{figure}
}

\definecolor{dkgreen}{rgb}{0,0.6,0}
\definecolor{gray}{rgb}{0.5,0.5,0.5}
\definecolor{mauve}{rgb}{0.58,0,0.82}


\lstset{frame=tb,
	language=Python,
	aboveskip=3mm,
	belowskip=3mm,
	showstringspaces=false,
	columns=flexible,
	basicstyle={\small\ttfamily},
	numbers=none,
	numberstyle=\tiny\color{gray},
	keywordstyle=\color{blue},
	commentstyle=\color{dkgreen},
	stringstyle=\color{mauve},
	breaklines=true,
	breakatwhitespace=true,
	tabsize=3
}

\begin{document}
	\selectlanguage{russian}
	

	\fefutitlepage{Б9119-02.03.01сцт}{Панченко Н.К.}{02}{июня}{22}
	
	
	\newpage
	
	\tableofcontents   
	\clearpage
	\section*{Введение}
	\addcontentsline{toc}{section}{Введение}
	Отчёт по лабораторной работе на тему <<Число обусловленности>>.	
	\newpage









	\section*{Число обусловленности}
	\addcontentsline{toc}{section}{Число обусловленности}
	Вычислить число обусловленности системы:
	\[\begin{cases}
		2x + y = 2\\
		(2-\varepsilon)x+y=1
	\end{cases}\]
	\section*{Решение}
	\[\begin{cases}
		y = 2- 2x\\
		2x- \varepsilon x + 2 -2x = 1
	\end{cases}\]
\[\varepsilon x = 1~~~ x = \dfrac{1}{\varepsilon}~~~y = 2 - \dfrac{2}{\varepsilon}\]
	\[||A||_1 = 2 + |2-\varepsilon|\]
	\[|A| = 2 -2 + \varepsilon = \varepsilon\]
	\[M = \begin{pmatrix}
		1 & -1\\
		-2+\varepsilon & 2
	\end{pmatrix}\]
	\[A^{-1} = \dfrac{1}{\varepsilon}\begin{pmatrix}
		1 & -1\\
		\varepsilon - 2 & 2
	\end{pmatrix}\]
	\[||A^{-1}||_1 = \begin{cases}
		\dfrac{3}{\varepsilon}, \varepsilon \leq 4\\
		\dfrac{-1 + \varepsilon}{\varepsilon}, \varepsilon > 4\varepsilon 
	\end{cases}\]
	\[\mu (A) = \begin{cases}
		\dfrac{6+3|2-\varepsilon|}{\varepsilon}, \varepsilon \leq \\
		\dfrac{(2+|2-\varepsilon|)(-1+\varepsilon)}{\varepsilon}, \varepsilon>4
	\end{cases}\]

\[\mu = \dfrac{6+3|2+\varepsilon|}{\varepsilon} = \begin{cases}
	\dfrac{2-3\varepsilon}{\varepsilon}, \varepsilon \leq 2\\
	3, \varepsilon > 2
\end{cases}\]
\[\dfrac{(-1+\varepsilon)(2+|2-\varepsilon|)}{\varepsilon} = (-1 + \varepsilon)\]
\[\mu(A) = \begin{cases}
	\dfrac{12 - 3\varepsilon}{\varepsilon}, \varepsilon \leq 2\\
	3, 2<\varepsilon \leq 4\\
	\varepsilon - 1,\varepsilon > 4.
\end{cases}\]
\end{document}