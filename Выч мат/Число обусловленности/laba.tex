\documentclass[14pt, titlepage,fleqn]{extarticle}
\usepackage[T1,T2A]{fontenc}
\usepackage[utf8]{inputenc}

\usepackage{amsmath}
\usepackage[russian]{babel}

\usepackage{titlepage}
\usepackage[final]{pdfpages}
\usepackage{listings}
\usepackage{color}
\usepackage{graphicx}
\usepackage{float} 

\usepackage{caption}


\newcommand{\InsertGraf}[2]{
	\begin{figure}[H]
		\center{\includegraphics[width = 1\textwidth]{#1}}
		\caption{#2}
	\end{figure}
}

\definecolor{dkgreen}{rgb}{0,0.6,0}
\definecolor{gray}{rgb}{0.5,0.5,0.5}
\definecolor{mauve}{rgb}{0.58,0,0.82}


\lstset{frame=tb,
	language=Python,
	aboveskip=3mm,
	belowskip=3mm,
	showstringspaces=false,
	columns=flexible,
	basicstyle={\small\ttfamily},
	numbers=none,
	numberstyle=\tiny\color{gray},
	keywordstyle=\color{blue},
	commentstyle=\color{dkgreen},
	stringstyle=\color{mauve},
	breaklines=true,
	breakatwhitespace=true,
	tabsize=3
}

\begin{document}
	\selectlanguage{russian}
	

	\fefutitlepage{Б9119-02.03.01сцт}{Панченко Н.К.}{02}{июня}{22}
	
	
	\newpage
	
	\tableofcontents   
	\clearpage
	\section*{Введение}
	\addcontentsline{toc}{section}{Введение}
	Отчёт по лабораторной работе на тему <<Число обусловленности>>.	
	\newpage









	\section*{Число обусловленности}
	\addcontentsline{toc}{section}{Число обусловленности}
	Числом обусловленности матрицы $A$ называется
	\[\mu(A) = \underset{x \neq 0,y\neq0}{\sup} \left(\dfrac{||Ax||}{||x||} / \dfrac{||Ax||}{||x||}\right)\]
	Чтобы понять смысл этой характеристики, Рассмотрим систему уравнений
	\[Ax = f ~~~~~~~~~~~~~~~~(2)\]
	С невырожденной матрицей $A$ и рассмотрим систему уравнений с возмщенной правйо частью
	\[A(x+\xi ) = f + \eta ~~~~~~~~~~~~~~~~(3)\]
	
	$\xi  - $ возмущение решени, вызванное возмещением $\eta$ правой части 
	Вычтем $(2)$ из $(3)$ имеем $A\xi  = \eta$
	Оценим:
	\[\dfrac{||\xi||}{||x||} \cdot \dfrac{||f||}{||\eta||} = \dfrac{||\xi||}{||x||} \cdot \dfrac{||Ax||}{||A\xi||} = \dfrac{||Ax||}{||x||}/ \dfrac{||A\xi||}{||\xi||}\leq \mu(A)\dfrac{||\eta||}{||f||};\]
	Отсюда:
	\[\dfrac{||\xi||}{||x||} \leq \mu(A)\dfrac{||\eta||}{||f||}. ~~~~~~~~~~~~~~~~(4)\]
	Заметим, что из определения точной верхне2 грани непосредственно следует, что $\mu(A) - $ наименьшая их констант, для которых при всех $f \neq 0$ и всех $\mu \neq 0$  справедливо $(4)$.\\
	Величины:
	\newpage
	\[\dfrac{||\mu||}{||f||} \text{- наз. отсносительным возмущениес правой части}\] 
	\[\dfrac{||\mu||}{||f||} \text{- отсносительным возмещением решения.}\] 
	Неравенство $(4)$ показывает, что если число $\mu(A)$ велико, то даже при мальньком относительном возмущении правой части относительное возмущение решения может быть достаточно большим: последнее может быть боьльше первого в $\mu(A)$ раз. Инными словами, при большом числе обусловленности даже малое изменение правой части может привести к большому изменению или плохо обусловлено ее правой частью.\\
	Если же число $\mu(A)$ невилико, то из $(4)$ слеждует, что малое изменение правой части приводит к малому изменению решения. В этом случае говорят, что решение системы хорошо обусловленно ее правой частью. \\
	Мерой обусловленности решения системы ее правой частью, как мы видим, и является число обусловленности матрицы этой системы.\\
	Матрицы с большим числом обусловленности называют плохо обусловленными, а матрицы, имеющие большое число обусловленности - хорошо обусловленными.\\
	Предполагаем, что $A$ невырожденная, получим формулу для $\mu(A):$
	\[\mu(A) = \dfrac{\underset{x \neq 0}{\sup} \dfrac{||Ax||}{||x||}}{\underset{y \neq 0}{\inf} \dfrac{||Ay||}{||y||}}\] 
	Числитель есть $||A||$ подчиненная норме векторной$||x||$\\
	Сделаем замену $x = Ay$






	\newpage
	\[\underset{y \neq  0}{\inf} \dfrac{||Ay||}{||y||} = \underset{z \neq  0}{\inf} \dfrac{||z||}{||A^{-1}z||} = \underset{z \neq  0}{\inf} \dfrac{1}{\dfrac{||A^{-1}z||}{||z||}} = \dfrac{1}{\underset{z \neq 0}{\sup}\dfrac{||A^{-1}z||}{||z||}} = \dfrac{1}{||A^{-1}||}\]
	Окончательно, $\mu(A) = ||A||\cdot ||A^{-1}||$\\
	Если , $A$- вырожденная, то однородная система $Ay = 0$ имеет нетривиальное решение и следовательно 
	\[\underset{y \neq 0} {\inf} \dfrac{||Ay||}{||y||} = 0\]
	В этом случае $\mu(A) = \infty$\\
	Отметим , что число обусловленности матрицы не может быть $\leq 1.$
	Выясняя смысл числа обусловленности $\mu(A)$ мы видели, что оно характеризует обусловленность решения системы $Ax = f$ ее праой частью. Оказывается, что $\mu(A)$ указывается и на характер обусловленности решения системы ее матрицей.\\
	Наряду с системой $Ax=f$ рассмотрим систему
	\[(A + \sigma)(x + \xi) = f + \eta\]
	Здесь $\xi - $ возмущение решения, вызванное возмущением $\eta$ правой части и возмущением $\sigma$ матрицы системы.
 	
	



	% Вычислить число обусловленности системы:
	% \[\begin{cases}
	% 	2x + y = 2\\
	% 	(2-\varepsilon)x+y=1
	% \end{cases}\]

	% \section*{Теория}
	% Рассмотрим линейное уравнение
	% \[Ax = b\]
	% где $A$ -- линейный оператор, $b$-- вектор, $x$ -- искомый вектор (переменная уравнения). Допустим, уравнение решается с погрешностью на входных данных $b\pm \bigtriangleup b$. Отношение относительных ошибок аргумента $b$ и решения $A^{-1}$ равно 
	% \[\dfrac{\dfrac{||A^{-1}\bigtriangleup b||}{||A^{-1}b||}}{\dfrac{||\bigtriangleup b||}{||b||}} = \left(\dfrac{||A^{-1}\bigtriangleup b||}{\bigtriangleup b}\right) \cdot \left(\dfrac{||b||}{||A^{-1} b||}\right).\]
	% Тогда число обусловленности $\mu(A) $характеризует, насколько велика будет погрешность решения при произвольных ненулевых $b$ и $e$.
	% \[\mu(A) = \underset{\bigtriangleup b, b\neq 0}{\max}\left\{\left(\dfrac{||A^{-1}\bigtriangleup b||}{||\bigtriangleup b||}\right) \cdot \left(\dfrac{||b||}{||A^{-1}b||}\right)\right\} = \underset{\bigtriangleup b\neq 0}{\max} \left\{\dfrac{||A^{-1}\bigtriangleup b||}{||\bigtriangleup b||} \right\} \cdot \]
	% \[\cdot  \underset{b\neq 0}{\max} \left\{\dfrac{||b||}{||A^{-1}b||} \right\} = \underset{\bigtriangleup b\neq 0}{\max} \left\{\dfrac{||A^{-1}\bigtriangleup b||}{||\bigtriangleup b||} \right\} \cdot \underset{x\neq 0}{\max} \left\{\dfrac{||Ax||}{||x||} \right\} = ||A^{-1}|| \cdot ||A||\]

	% \newpage
	% Такое же определение дается для любой операторной нормы (то есть определение зависит от выбора нормы):
	% \[\mu(A) = ||A|| \cdot ||A^{-1}||.\]
	% Если оператор $||A^{-1}||$ не ограничен, то числом обусловленности оператора $A$ обычно считают $\mu(A) = +\infty.$


	% С числом обусловленности связано множество утверждений и оценок теории вычислительной математики.

	% Если число обусловленности оператора $A$ мало, то оператор называется хорошо обусловленным. Если же число обусловленности велико, то оператор называется плохо обусловленным. Таким образом, чем меньше $\mu(A)$, тем «лучше», то есть тем меньше погрешности решения будут относительно погрешностей в условии. Учитывая, что $\mu(A) \geq 1 $, то наилучшим числом обусловленности является 1. 	

	\section*{Решение}
	Вычислить число обусловленности системы:
	\[\begin{cases}
		2x + y = 2\\
		(2-\varepsilon)x+y=1
	\end{cases}\]
	\[\begin{cases}
		y = 2- 2x\\
		2x- \varepsilon x + 2 -2x = 1
	\end{cases}\]
\[\varepsilon x = 1~~~ x = \dfrac{1}{\varepsilon}~~~y = 2 - \dfrac{2}{\varepsilon}\]
	\[||A||_1 = 2 + |2-\varepsilon|\]
	\[|A| = 2 -2 + \varepsilon = \varepsilon\]
	\[M = \begin{pmatrix}
		1 & -1\\
		-2+\varepsilon & 2
	\end{pmatrix}\]
	\[A^{-1} = \dfrac{1}{\varepsilon}\begin{pmatrix}
		1 & -1\\
		\varepsilon - 2 & 2
	\end{pmatrix}\]
	\[||A^{-1}||_1 = \begin{cases}
		\dfrac{3}{\varepsilon}, \varepsilon \leq 4\\
		\dfrac{-1 + \varepsilon}{\varepsilon}, \varepsilon > 4
	\end{cases}\]
	\[\mu (A) = \begin{cases}
		\dfrac{6+3|2-\varepsilon|}{\varepsilon}, \varepsilon \leq 4\\
		\dfrac{(2+|2-\varepsilon|)(-1+\varepsilon)}{\varepsilon}, \varepsilon>4
	\end{cases}\]

\[\mu = \dfrac{6+3|2+\varepsilon|}{\varepsilon} = \begin{cases}
	\dfrac{2-3\varepsilon}{\varepsilon}, \varepsilon \leq 2\\
	3, \varepsilon > 2
\end{cases}\]
% \[\dfrac{(-1+\varepsilon)(2+|2-\varepsilon|)}{\varepsilon} = (-1 + \varepsilon)\]
\[\mu(A) = \begin{cases}
	\dfrac{12 - 3\varepsilon}{\varepsilon}, 0 < \varepsilon \leq 2\\
	3, 2<\varepsilon \leq 4\\
	\varepsilon - 1,\varepsilon > 4.
\end{cases}\]
\end{document}