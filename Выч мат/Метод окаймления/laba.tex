\documentclass[14pt, titlepage,fleqn]{extarticle}
\usepackage[T1,T2A]{fontenc}
\usepackage[utf8]{inputenc}

\usepackage{amsmath}
\usepackage[russian]{babel}

\usepackage{titlepage}
\usepackage[final]{pdfpages}
\usepackage{listings}
\usepackage{color}
\usepackage{graphicx}
\usepackage{float} 

\usepackage{caption}


\newcommand{\InsertGraf}[2]{
	\begin{figure}[H]
		\center{\includegraphics[width = 1\textwidth]{#1}}
		\caption{#2}
	\end{figure}
}

\definecolor{dkgreen}{rgb}{0,0.6,0}
\definecolor{gray}{rgb}{0.5,0.5,0.5}
\definecolor{mauve}{rgb}{0.58,0,0.82}


\lstset{frame=tb,
	language=Python,
	aboveskip=3mm,
	belowskip=3mm,
	showstringspaces=false,
	columns=flexible,
	basicstyle={\small\ttfamily},
	numbers=none,
	numberstyle=\tiny\color{gray},
	keywordstyle=\color{blue},
	commentstyle=\color{dkgreen},
	stringstyle=\color{mauve},
	breaklines=true,
	breakatwhitespace=true,
	tabsize=3
}

\begin{document}
	\selectlanguage{russian}
	

	\fefutitlepage{Б9119-02.03.01сцт}{Панченко Н.К.}{02}{июня}{22}
	
	
	\newpage
	
	\tableofcontents   
	\clearpage
	\section*{Введение}
	\addcontentsline{toc}{section}{Введение}
	Отчёт по лабораторной работе на тему <<Метод окаймления>>.	
	\newpage









	\section*{Метод окаймления}
	\addcontentsline{toc}{section}{Метод окаймления}
	Изучить, понять и реализовать алгоритм метода оптимального ис-
	ключения для решения СЛАУ, а также описать работу алгоритма и
	привести результаты.

	\section*{Алгоритм}
	Введем обозначения:
	\[U_n = (a_{1n},...,a_{n-1,n}), ~~~~~ V_n =(a_{n1},...,a_{n,n-1})\]
	Для размерности $k$:
	\[a_k = a_{kk} - V_kA^{-1}_{k-1}U_k\]
	\[Q_k = - \dfrac{V_kA^{-1}_{k-1}}{a_k}\]
	\[P_{k-1} = A^{-1}_{k-1} - A^{-1}_{k-1}U_kQ_k\]
	\[R_k = - \dfrac{A^{-1}_{k-1}U_k}{a_k}\]
	\[A^{-1}_k = \begin{pmatrix}
		P_{k-1} & R_k\\
		Q_k & \dfrac{1}{a_k}
	\end{pmatrix}\]
	
	\section*{Тесты}

\end{document}