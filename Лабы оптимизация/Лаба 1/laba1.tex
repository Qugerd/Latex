\documentclass[14pt, titlepage,fleqn]{extarticle}
\usepackage[T1,T2A]{fontenc}
\usepackage[utf8]{inputenc}

\usepackage{amsmath}
\usepackage[russian]{babel}

\usepackage{titlepage}
\usepackage[final]{pdfpages}
\usepackage{listings}
\usepackage{color}
\usepackage{graphicx}
\usepackage{float} 

\usepackage{caption}


\newcommand{\InsertGraf}[2]{
	\begin{figure}[H]
		\center{\includegraphics[width = 1\textwidth]{#1}}
		\caption{#2}
	\end{figure}
}

\definecolor{dkgreen}{rgb}{0,0.6,0}
\definecolor{gray}{rgb}{0.5,0.5,0.5}
\definecolor{mauve}{rgb}{0.58,0,0.82}


\lstset{frame=single,
	language=Python,
	aboveskip=3mm,
	belowskip=3mm,
	showstringspaces=false,
	columns=flexible,
	basicstyle={\small\ttfamily},
	numbers=none,
	numberstyle=\tiny\color{gray},
	keywordstyle=\color{blue},
	commentstyle=\color{dkgreen},
	stringstyle=\color{mauve},
	breaklines=true,
	breakatwhitespace=true,
	tabsize=3
}

\begin{document}
	\selectlanguage{russian}
	

	\fefutitlepage{Б9119-02.03.01сцт}{Панченко Н.К.}{17}{мая}{22}
	
	
	\newpage
	
 
	\clearpage
	\section*{Постановка задачи}
	Требуется найти минимум функции $f(x) = \dfrac{1}{2}x^TAx + bx$ нескольких переменных при помощи методов Ньютона и градиентного спуска и сравнить результаты и время работы.\\
	Для начала нужно сгенерировать симметричную и положительно определенную матрицу А. Сгенерировав нижнюю треугольную матрицу L, где все элементы на главной диагонали строго больше нуля, и умножив ее на транспонированную матрицу L мы получим необходимую симметрическую и положительно определенную матрицу. В правую часть можно записать любые числа.\\
	В обоих методах необходима производная функции, так же в методе Ньютона используется вторая производная.
	\[f(x)=\dfrac{1}{2} x^T Ax+bx\]
	\[f'(x) = \left( \dfrac{1}{2} x^T Ax  \right)' + (bx)' = \dfrac{1}{2}\left(A+ A^T \right)x+b=Ax+b\]
	\[f''(x)= (Ax)' + b' = A\]
	Для метода Ньютона необходимо нахождение обратной матрицы. Будем искать обратную матрицу с помощью матрицы из алгебраических дополнений.

	\begin{lstlisting}
		def det(a):
	if(type(a[0]) != list or len(a) != len(a[0])):
		return None
	if(len(a) == 2):
		return a[0][0] * a[1][1] - a[0][1]*a[1][0]
	elif(len(a) == 3):
		sum = 0
		for i in range(3):
			p = 1
			m = 1
			for j in range(3):
				p *= a[(i + j)%3][j]
				m *= a[(i + j)%3][2-j]
			sum += p
			sum -= m
		return sum
	else:
		sum = 0
		for i in range(len(a)):
			sum += (-1)**i * a[0][i]*det(minor(a,i,0))
		return sum

def minor(a,x,y):
	to_ret = []
	for r in a:
		temp = r[:]
		temp.pop(x)
		to_ret += [temp]
	to_ret.pop(y)
	return to_ret

def inv(a):
	c = [r[:] for r in a]
	for i in range(len(a)):
		for j in range(len(a[0])):
			c[i][j] = (-1)**(i + j)*det(minor(a,j,i))
	return mul(transpose(c),1/det(a))


	\end{lstlisting}
	\newpage


	\section*{Метод Ньютона}
	Поиск значений x для нахождения минимума функции осуществляется по следующей формуле:
	\[x^{k+1} = x^k - \dfrac{f'(x^k)}{f''(x^k)}\]
	В нашем случае начальные значения x не влияют на ход работы и для вычисления значений x потребуется лишь одна итерация, потому как:
	\[x^1 = x^0 - \dfrac{f'(x^0)}{f''(x^0)}= x^0 - A^{-1}(Ax^0 + b) = x^0 - A^{-1}Ax^0 - A^{-1}b = - A^{-1}b\]
	Тогда подставив в производную найденные значения получим, что мы нашли стационарную точку, являющуюся минимумом функции:
	\[f'(x^1) = Ax^{1} + b = A(-A^{-1}b) + b = -b +b =0\]
	Вычисления происходили бы не за один шаг, если бы у нас была другая функция, соответственно другие производные.
	\section*{Реализация алгоритма}
	Для реализации воспользуемся языком программирования Python и библиотекой Numpy для удобной работы с матрицами.
	\begin{lstlisting}[title=dwdw, captionpos=b]
		def newtons_method(x,a,b):
	return x - dot(inv(a), dfunc(x,a,b))
	\end{lstlisting}
	
	\section*{Метод градиентного спуска}
	Поиск значений x для нахождения минимума функции осуществляется по следующей формуле:
	\[x^{k+1} = x^k - \lambda f'(x^k)\]
	Будем брать $\lambda = 10^{-6}$, а вычисления проводить до тех пор, пока разница между значениями функций $f(x^k)$ и $f(x^{k+1})$ будет больше, чем $10^{-10}$.


	\section*{Реализация алгоритма}
	Для реализации воспользуемся языком программирования Python и библиотекой Numpy для удобной работы с матрицами.
	\begin{lstlisting}[title=dwdw, captionpos=b]
		def grad_method(x,a,b,step = 0.0001):
	xn = [r[:] for r in x]
	while(step > 0.000000001):
		start = func(xn,a,b)
		d = dfunc(xn,a,b)
		right  = mul(d ,-step)
		xn = plus(xn, right)
		end = func(xn,a,b)
		if(start < end):
			step /= 10
	return xn

	\end{lstlisting}





	\newpage
	\section*{Тесты}



	\newpage
	\section*{Заключение}
\end{document}