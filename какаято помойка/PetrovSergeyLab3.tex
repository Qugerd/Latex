\documentclass[14pt, titlepage,fleqn]{extarticle}
\usepackage[T1,T2A]{fontenc}
\usepackage[utf8]{inputenc}

\usepackage{amsmath}
\usepackage[russian]{babel}

\usepackage{titlepage}
\usepackage[final]{pdfpages}
\usepackage{listings}
\usepackage{color}
\usepackage{graphicx}
\usepackage{float} 

\usepackage{caption}


\newcommand{\InsertGraf}[2]{
	\begin{figure}[H]
		\center{\includegraphics[width = 1\textwidth]{#1}}
		\caption{#2}
	\end{figure}
}

\definecolor{dkgreen}{rgb}{0,0.6,0}
\definecolor{gray}{rgb}{0.5,0.5,0.5}
\definecolor{mauve}{rgb}{0.58,0,0.82}


\lstset{frame=tb,
	language=Python,
	aboveskip=3mm,
	belowskip=3mm,
	showstringspaces=false,
	columns=flexible,
	basicstyle={\small\ttfamily},
	numbers=none,
	numberstyle=\tiny\color{gray},
	keywordstyle=\color{blue},
	commentstyle=\color{dkgreen},
	stringstyle=\color{mauve},
	breaklines=true,
	breakatwhitespace=true,
	tabsize=3
}

\begin{document}
	\selectlanguage{russian}
	

	%\fefutitlepage{Б9119-02.03.01сцт}{Панченко Н.К.}{2}{апреля}{22}
	
	
	\newpage
	
	%\tableofcontents   
	\clearpage
	% \section*{Введение}
	% \addcontentsline{toc}{section}{Введение}
	\section*{Вариант 1}
	1. Доказательство вычислительной формулы  подчиненной  матричной  нормы $||A||_2$.\\
	2. Определение подчиненной и согласованной матричных норм, их взаимосвязь, свойства подчиненной матричной нормы.\\
	3. LU - разложение.


	\section*{Вариант 2}
	1. Доказательство вычислительной формулы  подчиненной  матричной  нормы $||A||_\infty$.\\
	2. Обусловаленность матриц и систем. Определение, вывод вычислительной формулы.\\	
	3. Необходимое и достаточное условие LU - разложения (доказательсво).



	
	
	\newpage
	
	

\end{document}